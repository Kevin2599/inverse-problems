\chapter{Regularização}

As equações \ref{eq:sistema_normal} do problema inverso linear e
\ref{eq:sistema_normal_gaussnewton} do problema inverso não-linear são
{\it equações normais}.
Estas equações são sistemas lineares cuja matriz é quadrada.
Para que a solução de uma equação normal seja única, é necessário que esta
matriz tenha {\it posto completo}.
Uma condição para que uma matriz tenha posto completo é que todas as suas colunas
(ou linhas) sejam {\it linearmente independentes}.
Esta condição é equivalente a dizer que a matriz possui determinante diferente de
zero.
\\
\indent Em problemas geofísicos, é comum que a matriz do sistema normal possua
determinante próximo de zero. Ou seja, para fins práticos pode-se considerar que
a matriz {\it não} tenha posto completo.
Isto faz com que o problema inverso geofísico seja um problema {\it mal posto}.

\begin{quote}
{\tt Um problema {\bf mal-posto} apresenta, principalmente, {\it instabilidade}
e {\it falta de unicidade} da solução.}
\end{quote}

\begin{quote}
{\tt A {\bf instabilidade} é a alta variabilidade dos parâmetros perante
peque\-nas variações nos dados.}
\end{quote}

\begin{example}
Seja um conjunto de dados preditos A e outro conjunto de
dados preditos ligeiramente diferentes B.
Se o problema inverso apresenta instabilidade, os pa\-râ\-me\-tros que produzem os
dados preditos A são consideravelmente diferentes daqueles que produzem os dados
preditos B.
\end{example}

\begin{quote}
{\tt A {\bf falta de unicidade} é a existência de vários conjuntos de parâme\-tros
que produzem os mesmos dados preditos.}
\end{quote}

\noindent Se o problema inverso apresenta falta de unicidade,
os dados observados podem ser explicados por vários
conjuntos de parâmetros diferentes. Em termos práticos, existem vários vetores
de parâmetros diferentes que minimizam a função do ajuste
(equação \ref{eq:ajuste}).
\\
\indent Os problemas inversos encontrados na geofísica são, em geral, mal-postos.
Os principais motivos para isto são: a presença de ruído nos dados observados; e,
principalmente, a própria natureza do problema inverso.
Neste último caso, mesmo se fôssemos capazes de obter dados completamente isentos de ruído, ainda
teríamos diversas combinações de parâmetros capazes de explicar os dados
observados (i.e., falta de unicidade).
Por este motivo, quando nos deparamos com problemas inversos na geofísica,
é essencial a utilização de {\it regularização}.
\\
\indent A {\it regularização} é um procedimento matemático que contorna os
problemas de ins\-tabilidade e falta de unicidade em problemas inversos mal-postos.
Esse procedimento equivale a impor restrições aos parâmetros a serem estimados.
Desta forma, em um {\it problema inverso regularizado} buscamos estimar um
conjunto de parâmetros que ajustam os dados observados e satisfaçam
determinadas restrições.
Estas restrições introduzem informações {\it a priori} no problema inverso.
As informações podem ser de natureza geológica ou meramente matemática.
Em geral, a introdução de informações a priori é feita por meio de {\it funções
regularizadoras}, que são funções escalares que dependem dos parâmetros.
Para incorporar informação a priori ao ajuste dos dados, formamos a
{\it função objetivo}

\begin{equation}
\Omega(\vect{p}) = \phi(\vect{p}) + \mu\theta(\vect{p}) \thinspace ,
\label{eq:objetivo}
\end{equation}

\noindent em que $\phi(\vect{p})$ é a função do ajuste (equação \ref{eq:ajuste}),
$\theta(\vect{p})$ é uma função regularizadora e $\mu$ é um escalar positivo
denominado {\it parâmetro de regularização}.
Desta forma, o {\it problema inverso regularizado} é definido como estimar um
vetor de parâmetros $\opt{p}$ que {\it minimiza a função objetivo}.
\\
\indent Neste ponto é importante ressaltar que

\begin{quote}
{\tt O {\bf parâmetro de regularização} $\mu$ controla a importância relativa
en\-tre o ajuste aos dados observados e a concordância com a infor\-ma\-ção
a priori.}
\end{quote}

\noindent O valor de $\mu$ é definido pelo usuário da inversão e por isso é
importante compreender sua influência no resultado obtido (parâmetros estimados).
Valores altos de $\mu$ tornam o problema inverso bem posto e fazem com que os
parâmetros estimados satisfaçam quase completamente as informações a priori.
Porém, isto geralmente faz com que haja um desajuste entre os dados observados e
preditos.
Por outro lado, valores baixos de $\mu$ fazem com que os parâmetros estimados
ajustem os dados observados. No entanto, a estimativa poderá ser não-única e/ou
instável, dependendo de quanto o problema inverso for mal-posto.
Idealmente, deve-se encontrar um valor de $\mu$ que proporcione um bom ajuste e
satisfaça as informações a priori o suficiente para estabilizar a solução.
Procedimentos práticos para determinação do valor de $\mu$ serão discutidos no
Capítulo \ref{chap:proc_praticos}.
\\
\indent No problema inverso regularizado, a linearidade não depende apenas da
função $f_i(\vect{p})$ (equação \ref{eq:fi}) que relaciona os parâmetros aos
dados preditos, mas também da função regularizadora.
Nesse caso, para que o problema inverso seja linear, é necessário que tanto
$f_i(\vect{p})$ como o {\it gradiente da função regularizadora} sejam
combinações lineares dos parâmetros.
Se ao menos uma destas funções for não-linear, o problema inverso deverá ser
resolvido como um problema inverso não-linear.
\\
\indent Como foi observado anteriormente (Seção \ref{sec:nao-linear}), o
problema inverso linear é um caso particular do problema inverso não-linear.
Por esta razão, a formulação geral para o problema inverso regularizado
será feita seguindo os procedimentos adotados para o problema inverso não-linear.
Assim sendo, iniciaremos expandindo a função objetivo (equação
\ref{eq:objetivo}) em série de Taylor até segunda ordem

\begin{equation}
\Omega(\vect{p}_0 + \Delta\vect{p}) \approx \Omega(\vect{p}_0) +
    \vect{\nabla}\Omega(\vect{p}_0)^T\Delta\vect{p} +
    \frac{1}{2}\Delta\vect{p}^T\mat{\nabla}\Omega(\vect{p}_0)\Delta\vect{p}
    \thinspace ,
\label{eq:objetivo_taylor}
\end{equation}

\noindent sendo o vetor gradiente $\vect{\nabla}\Omega(\vect{p}_0)$ e a matriz
Hessiana $\mat{\nabla}\Omega(\vect{p}_0)$ de $\Omega(\vect{p})$ dados por

\begin{equation}
\vect{\nabla}\Omega(\vect{p}_0) = \vect{\nabla}\phi(\vect{p}_0) +
    \mu\vect{\nabla}\theta(\vect{p}_0) \thinspace
\label{eq:grad_objetivo}
\end{equation}

\noindent e

\begin{equation}
\mat{\nabla}\Omega(\vect{p}_0) = \mat{\nabla}\phi(\vect{p}_0) +
    \mu\mat{\nabla}\theta(\vect{p}_0) \thinspace ,
\label{eq:hessian_objetivo}
\end{equation}

\noindent em que $\mu$ é o parâmetro de regularização,
$\vect{\nabla}\phi(\vect{p}_0)$ e $\mat{\nabla}\phi(\vect{p}_0)$
são o gradiente e a Hessiana da função do ajuste (equações \ref{eq:gradphi} e
\ref{eq:hessian_approx}) e
$\vect{\nabla}\theta(\vect{p}_0)$ e $\mat{\nabla}\theta(\vect{p}_0)$ são o
gradiente e a Hessiana da função regularizadora, todos avaliados em $\vect{p}_0$.
\\
\indent Seguindo a dedução feita para o problema inverso não-linear (Seção
\ref{sec:nao-linear}), a correção $\Delta\vect{p}$ em uma determinada iteração
do método Gauss-Newton é

\begin{equation}
\mat{\nabla}\Omega(\vect{p}_0)\Delta\vect{p} = -\vect{\nabla}\Omega(\vect{p}_0)
    \thinspace .
\label{eq:sistema_normal_objetivo}
\end{equation}

\noindent Substituindo o gradiente e a Hessiana da função do ajuste (equações
\ref{eq:gradphi} e \ref{eq:hessian_approx}) obtemos

\begin{equation}
\left[2\mat{G}(\vect{p}_0)^T\mat{G}(\vect{p}_0) +
      \mu\mat{\nabla}\theta(\vect{p}_0)\right]\Delta\vect{p} =
2\mat{G}(\vect{p}_0)^T \left[\vect{d}^{\thinspace o} - \vect{f}(\vect{p}_0)\right] -
\mu\vect{\nabla}\theta(\vect{p}_0)
    \thinspace .
\label{eq:sistema_normal_regul}
\end{equation}

\indent Esta é a equação para o {\it problema inverso regularizado}.
\footnote{Tente obter a equação normal para o caso em que $\vect{f}(\vect{p})$ é linear.}
A seguir, apresentaremos diferentes funções regularizadoras comumente encontradas
em problemas inversos na geofísica.
Também mostraremos como fica a equação \ref{eq:sistema_normal_regul} para cada
um dos casos.

